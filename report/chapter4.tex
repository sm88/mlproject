\chapter{Final Submission}
\section{Problem Statement}
\label{prob-stmt}
Detecting emotional state of a person by analyzing a text document
written by him appear challenging but it is also essential many times. Recognizing the emotion of the text plays a key role in 
the human-computer interaction. Emotions may be expressed by a person's speech, 
face  expression  and  written  text  known  as  speech,  facial  and  text  based  emotion 
respectively. Athough sufficient amount of work has been done by researchers to detect emotion from facial and 
speech information but recognizing  emotions  from  textual  data  is still a fresh and hot 
research area. In this project, we have addressed the problem of emotion detection on the basis of text data.
\subsection{Why is it interesting}
\begin{itemize}
\item Psychologists can better assist their patients by analyzing their session transcripts for any subtle emotions.
\item Reliable emotion detection can help develop powerful human-computer interaction devices. For example,a computer can analysis the emotion during a text chat session and according display proper emoticons or a computer can read a movie review and then analyze the sentiment of the reviewer.
\item Deep  emotional  analysis  of  public  data such  as  tweets  and  blogs  could  reveal  interesting  insights into human nature and behavior
\end{itemize}
\section{Ideas Used From Class}
\label{ideas-from-class}
\begin{itemize}
    \item Apart from Vector Space Model, we have also used SVM and gaussian naive bayes classifier to solve this classification problem.
    \item We have measured the overall F1 score of the prediction to compare the
        performance of different classification algorithms.
\end{itemize}

\section{Concepts Learned in this project}
\begin{itemize}
    \item Learnt Vector Space model which is an algebrical model for representing
        text document as a vector of index terms. This is an widely used
        technique in information retrieval.
    \item Although we studied SVM and Gaussian naive bayes classifier in our
        class, in this project we learnt how to employ them in a practical
        problem like detection of emotion from text.
    \item During the course of this project, we also became familiar with
        tf-idf (term frequency-inverse document frequency) weighting scheme.
        This is an well known technique in information retrieval which gives us
        an idea about the importance of an word to a document in a corpus.
    \item We also had a hands on experience with Weka which is a popular suite
        of machine learning algorithms. We used this tool mainly to visualize the
        training data and to test some standard classification algorithm.
\end{itemize}

