\chapter{Submission 1 ($12^{th} March)$}
\section{Project Description}
\subsection{Problem Statement}
Text written by a person can help us determine what emotion he is portraying. Other such modalities can be facial cues, body language etc. In this project we will try to predict the emotion using information from a sentence input by a user. This can have applications in chatting programs, movie reviews etc. Consider the following scenarios:

\begin{table}[ht!]
\centering
\label{tab-some-examples}
\begin{tabular}{l|c}
\textbf{Sentence} & \textbf{Emotion} \\
\hline
$\bullet$A close person lied to me & Sadness \\
$\bullet$A colleague asked me for some advice and & \\
as he did not have enough confidence in me & \\
he asked a third person. & Sadness \\
$\bullet$I got a present today. & Joy
\end{tabular}
\caption{Some examples}
\end{table}

\subsection{Basic Details}
\label{subsec-basic-details}
\begin{itemize}
\item We wish to implement a system capable of predicting the emotion portrayed, given a sentence. The predicted classes will be limited to \textbf{anger, joy and sadness}.
\item The project is inspired from the paper mentioned in section \ref{sec-feasibility}.
\item The paper details the use of \textbf{Vector Space Models} (VSM) to implement the solution.
\item We will be using our dataset, comprising sentences annotated with emotions to train the model. The trained model will then be presented with the test data in form of a vector and the most probable emotion will be displayed as output.
\item Later, if time permits, we will implement some form of application to showcase the prowess of our project.
\end{itemize}
\section{Papers}
The paper implemented is mentioned in \ref{sec-feasibility}.
\section{Dataset}
The dataset used is mentioned in \ref{sec-dataset}. The dataset has around 7500 sentences. We will use 2/$3^{rd}$ of the dataset for training and cross validation and the rest for testing.
\section{Method}
\begin{itemize}
\item As mentioned in section \ref{subsec-basic-details}, in line with the paper, we will be using VSM for out initial implementation.
\item Put simply, we will have representative vectors for each of our emotion classes, \textbf{joy, anger and sadness}. Then we will transform our input sentence in the same way we found our representative vectors and find the cosine similarity with each of them.
\item These similarities will be used to calculate probabilities of the input being in some emotion class and the most probable class will be output.
\item The vectors will comprise of weights in terms of \textbf{tf-idf} frequencies. We could simply use word frequency instead but:
	\begin{itemize}
		\item \textbf{tf-idf} was chosen to provide weights as some words occur sparsely in the document, but provide definitive information about the emotion being portrayed. For example, the word \emph{devil} occurring in a sentence tips the scale towards negative emotion.
		\item Had we chosen only frequencies, this could pollute the result.
		\item Similarly, some words occur very frequently like \emph{cat}, but do not provide much information. Frequentistically speaking these words will get higher weight.
		\item \textbf{tf-idf} balances these disparities by multiplying the term frequency with inverse document frequency. The latter meaning the ratio of total number of documents and the number of documents containing that term.
		\item For low frequency words, the \textbf{idf} will be high and vice-versa.
	\end{itemize}
\end{itemize}
\subsection{High level algorithm}
\label{subsec-high-level-alg}
\textbf{Training}
\begin{itemize}
\item The training set is cleaned, in that:
	\begin{itemize}
	\item Stop words are removed, after comparing with standard stop word lists.
	\item Apostrophes are expanded.
	\item Words are reduced to their root using some stemming technique.
	\end{itemize}
\item We build a lexicon from 2/3$^{rd}$ of the above cleaned data.
\item Each term in the vector corresponds to a term in our lexicon. The lexicon is an ordered set of all words in the cleaned dataset.
\item We build document vectors using weights assigned to each term. These weights are calculated using the \textbf{tf-idf} technique. These weights are further normalized.
$$
w_{ki}=c(t_k,d_i)=\frac{tf_{ik}log(N/n_k)}{\sqrt{\sum_{k=1}^{n} (tf_{ik})^{2}[log(N/n_k)]^2}}
$$
where, \\
$t_k$ = $k^{th}$ term in document $d_i$ \\
$tf_{ik}$ = frequency of the word $t_k$ in document $d_i$ \\
$idf_k$ = log($\frac{N}{n_k}$), inverse document frequency of word $t_k$ in  entire dataset. \\
$n_k$ = number of documents with word $t_k$. \\
$N$ = total number of documents in the dataset.
\item Now each emotion class is a set of the corresponding vectors, calculated as above. We then find the mean of all these vectors to find a representative vector of the said class.
\item Finally, we'll get $s$ vectors corresponding to the distinct emotion classes.
\end{itemize}
\textbf{Testing}
\begin{itemize}
\item Each input sentence is cleaned, and transformed into a vector with weights calculated using \textbf{tf-idf} technique.
\item The similarity is calculated as:
$$
\mbox{sim}(Q,E_j) = \sum_{k=1}^{n} w_{kq}*E_{kj}
$$
where,\\
$Q$ is query vector, \\
$E_j$ is an emotion vector.
\item Answer is calculated as:
$$
\mbox{VSM}(Q) = \mbox{argmax}_{j} (\mbox{sim}(Q, E_j))
$$
\end{itemize}
\section{Submission on $21^{st}$}
We plan to complete the implementation of the paper along with relevant testing and output metrics, outlined above. We will complete the code for text cleaning and VSM.
\section{Final submission}
\begin{itemize}
\item Complete implementation of VSM.
\item Implement some bayesian model using bigram, trigram or ngram approach and check the variation in results.
\item If time permits, we'll develop some applications to showcase the model.
\end{itemize}
\section{Team Members}
\begin{table}[ht!]
\centering
\label{tab-team-members}
\begin{tabular}{c|c}
\textbf{Name} & \textbf{Roll No} \\
\hline
Ravi Shankar Mondal \textbf{(leader)}& 133059003 \\
Srikrishna Khedkar & 143050055 \\
Saurabh Jain & 143050048 \\
Sushant Mahajan & 133059007
\end{tabular}
\caption{Team members}
\end{table}

